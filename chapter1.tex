\chapter{Name of first chapter}

\section{First section}

Here is some text. Note that all of the sections and subsections will automatically be included in the table of contents. Cite your references in this manner \cite{anybody}. It is possible to emphasize text by making it \emph{italic} or \textbf{bold}.

\subsection{Subsection}

Numbering depth goes down to subsection. 

\subsubsection{Subsubsection}
Subsubsections are not numbered and do not appear in the table of contents by default. 


\section{Second section}

Figures (such as Figure \ref{Fig1}) are included like this. You'll probably want to load the \verb!graphicx! package to include images.

\begin{figure}[h]
 \caption{A description of the figure.}
 \label{Fig1}
\end{figure}


\subsection{Another subsection}

Tables (such as Table \ref{Tab1}) can be constructed and referenced like this. Footnotes\footnote{This is a footnote.} can be created like this. Make and reference numbered equations, such as equation \eqref{eq1}, like this
\begin{equation}
 \label{eq1}
 \int_{\Omega} \left(\frac{f(\gamma_1,z^{\mathcal{E}(\cos^5(x))})}{2^\ell\sqrt{g(\beta)}}\right)=\sum_{i=1}^\infty o(i\log y).
\end{equation}

\begin{definition}
 A definition is a statement of the exact meaning of a word.
\end{definition}


\begin{theorem}
 Theorems look like this. 
\end{theorem}
\begin{proof}
 And this is the proof.
\end{proof}

\begin{table}
\centering
\begin{tabular}{|c|l|}
\hline
First column & Second column\\\hline
$a$          & Latin \\
$\aleph$     & Hebrew \\
$\alpha$     & Greek \\
7            & $1.02232134\times10^{-8}$\\
Monday       & Quidditch\\
Adam         & Dishwashing\\
\hline
\end{tabular}
\caption{A description of the table.} 

\label{Tab1}
\end{table}

